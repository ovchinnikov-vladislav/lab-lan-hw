\documentclass[a4paper,12pt]{article}

\usepackage[utf8x]{inputenc}
\usepackage[T2A]{fontenc}
\usepackage[english, russian]{babel}

% Опционно, требует  apt-get install scalable-cyrfonts.*
% и удаления одной строчки в cyrtimes.sty
% Сточку не удалять!
% \usepackage{cyrtimes}

% Картнки и tikz
\usepackage{graphicx}
\usepackage{tikz}
\usetikzlibrary{snakes,arrows,shapes}


% Некоторая русификация.
\usepackage{misccorr}
\usepackage{indentfirst}
\renewcommand{\labelitemi}{\normalfont\bfseries{--}}

% Увы, поля придётся уменьшить из-за листингов.
\topmargin -1cm
\oddsidemargin -0.5cm
\evensidemargin -0.5cm
\textwidth 17cm
\textheight 24cm

\sloppy

% Оглавление в PDF
\usepackage[
bookmarks=true,
colorlinks=true, linkcolor=black, anchorcolor=black, citecolor=black, menucolor=black,filecolor=black, urlcolor=black,
unicode=true
]{hyperref}

% Для исходного кода в тексте
\newcommand{\Code}[1]{\textbf{#1}}

\usepackage{verbatim}
\usepackage{fancyvrb}
\fvset{frame=leftline, fontsize=\small, framerule=0.4mm, rulecolor=\color{darkgray}, commandchars=\\\{\}}
\renewcommand{\theFancyVerbLine}{\small\arabic{FancyVerbLine}}



\title{Отчёт по лабораторной работе \\ <<Локальные сети>>}
\author{Овчинников Владислав Александрович}

\begin{document}

\maketitle

\tableofcontents

% Текст отчёта должен быть читаемым!!! Написанное здесь является рыбой.

\section{Получение адреса по DHCP}

Где что дампим (дампить с -tenv -s 1000).

\begin{Verbatim}
получение "случайного" адреса
\end{Verbatim}

Где что дампим.

\begin{Verbatim}
получение "фиксированого" адреса
\end{Verbatim}


\section{Использование VPN}

\begin{Verbatim}
ip r на маршрутизаторе после VPN и работы RIP
\end{Verbatim}

\begin{Verbatim}
ip -4 a  на маршрутизаторе
\end{Verbatim}

\begin{Verbatim}
просшулка сообщений RIP на tun0
\end{Verbatim}

Проверка работы VPN

\begin{Verbatim}
Трейс с ws21 до s11
\end{Verbatim}

\section{Правила фильтации пакетов и трансляции пдресов}

Где что дампим. 

\begin{Verbatim}
сценарий фильтрации
\end{Verbatim}

\begin{Verbatim}
iptables -L -nv
\end{Verbatim}

\begin{Verbatim}
iptables -L -nv -t nat
\end{Verbatim}

\section{Проверка трансляции SNAT}

Где что дампим.

\begin{Verbatim}
дамп SNAT в LAN (как вариант -i any tcp)
\end{Verbatim}

\begin{Verbatim}
дамп SNAT (снаружи)
\end{Verbatim}


\section{Проверка правил фильтрации}

Используем telnet.

\section{Проверка доступа к внутреннему серверу}

Используем telnet / веб-браузер на реальной машине. 
Должен быть виден DNAT и разрешённый доступ.

\end{document}
