\documentclass[a4paper,12pt]{article}

\usepackage[utf8x]{inputenc}
\usepackage[T2A]{fontenc}
\usepackage[english, russian]{babel}

% Опционно, требует  apt-get install scalable-cyrfonts.*
% и удаления одной строчки в cyrtimes.sty
% Сточку не удалять!
% \usepackage{cyrtimes}

% Картнки и tikz
\usepackage{graphicx}
\usepackage{tikz}
\usetikzlibrary{snakes,arrows,shapes}


% Некоторая русификация.
\usepackage{misccorr}
\usepackage{indentfirst}
\renewcommand{\labelitemi}{\normalfont\bfseries{--}}

% Увы, поля придётся уменьшить из-за листингов.
\topmargin -1cm
\oddsidemargin -0.5cm
\evensidemargin -0.5cm
\textwidth 17cm
\textheight 24cm

\sloppy

% Оглавление в PDF
\usepackage[
bookmarks=true,
colorlinks=true, linkcolor=black, anchorcolor=black, citecolor=black, menucolor=black,filecolor=black, urlcolor=black,
unicode=true
]{hyperref}

% Для исходного кода в тексте
\newcommand{\Code}[1]{\textbf{#1}}

\usepackage{verbatim}
\usepackage{fancyvrb}
\fvset{frame=leftline, fontsize=\small, framerule=0.4mm, rulecolor=\color{darkgray}, commandchars=\\\{\}}
\renewcommand{\theFancyVerbLine}{\small\arabic{FancyVerbLine}}



\title{Отчёт по лабораторной работе \\ <<Локальные сети>>}
\author{Овчинников Владислав Александрович}

\begin{document}

\maketitle

\tableofcontents

% Текст отчёта должен быть читаемым!!! Написанное здесь является рыбой.

\section{Получение адреса по DHCP}

Для изучения работы протокола DHCP в случае выдачи случайного IP-адреса был запущен DHCP-сервер на маршрутизаторе \textbf{r2}. Данный сервер настроен так, чтобы клиенты получали динамические IP-адреса из диапазона, заданного в конфигурационном файле.

Конфигурационный файл DHCP-сервера на маршрутизаторе \textbf{r2}:
\begin{Verbatim}
subnet 172.16.0.0 netmask 255.255.0.0 {}

subnet 10.20.0.0 netmask 255.255.0.0
{
  range 10.20.0.2 10.20.0.200;
  option routers 10.20.0.1;
  option domain-name-servers 10.20.0.1;
}
\end{Verbatim}

Дамп собирался при помощи программы \textbf{tcpdump} на маршрутизаторе \textbf{r2}, запущенной с параметрами \textbf{-tenv -s 1000}.

\begin{Verbatim}
10:10:10:10:10:ee > ff:ff:ff:ff:ff:ff, ethertype IPv4 (0x0800), length 342: (tos 0x10, ttl 128, id 0, offset 0, flags [none], proto UDP (17), length 328) 0.0.0.0.68 > 255.255.255.255.67: BOOTP/DHCP, Request from 10:10:10:10:10:ee, length 300, xid 0x7b8f9a6c, Flags [none]
	  Client-Ethernet-Address 10:10:10:10:10:ee
	  Vendor-rfc1048 Extensions
	    Magic Cookie 0x63825363
	    DHCP-Message Option 53, length 1: Discover
	    Parameter-Request Option 55, length 12: 
	      Subnet-Mask, BR, Time-Zone, Default-Gateway
	      Domain-Name, Domain-Name-Server, Option 119, Hostname
	      Netbios-Name-Server, Netbios-Scope, MTU, Classless-Static-Route
3a:40:ee:31:9e:cd > 10:10:10:10:10:ee, ethertype IPv4 (0x0800), length 342: (tos 0x10, ttl 128, id 0, offset 0, flags [none], proto UDP (17), length 328) 10.20.0.1.67 > 10.20.0.2.68: BOOTP/DHCP, Reply, length 300, xid 0x7b8f9a6c, Flags [none]
	  Your-IP 10.20.0.2
	  Client-Ethernet-Address 10:10:10:10:10:ee
	  Vendor-rfc1048 Extensions
	    Magic Cookie 0x63825363
	    DHCP-Message Option 53, length 1: Offer
	    Server-ID Option 54, length 4: 10.20.0.1
	    Lease-Time Option 51, length 4: 43200
	    Subnet-Mask Option 1, length 4: 255.255.0.0
	    Default-Gateway Option 3, length 4: 10.20.0.1
	    Domain-Name-Server Option 6, length 4: 10.20.0.1
10:10:10:10:10:ee > ff:ff:ff:ff:ff:ff, ethertype IPv4 (0x0800), length 342: (tos 0x10, ttl 128, id 0, offset 0, flags [none], proto UDP (17), length 328) 0.0.0.0.68 > 255.255.255.255.67: BOOTP/DHCP, Request from 10:10:10:10:10:ee, length 300, xid 0x7b8f9a6c, Flags [none]
	  Client-Ethernet-Address 10:10:10:10:10:ee
	  Vendor-rfc1048 Extensions
	    Magic Cookie 0x63825363
	    DHCP-Message Option 53, length 1: Request
	    Server-ID Option 54, length 4: 10.20.0.1
	    Requested-IP Option 50, length 4: 10.20.0.2
	    Parameter-Request Option 55, length 12: 
	      Subnet-Mask, BR, Time-Zone, Default-Gateway
	      Domain-Name, Domain-Name-Server, Option 119, Hostname
	      Netbios-Name-Server, Netbios-Scope, MTU, Classless-Static-Route
3a:40:ee:31:9e:cd > 10:10:10:10:10:ee, ethertype IPv4 (0x0800), length 342: (tos 0x10, ttl 128, id 0, offset 0, flags [none], proto UDP (17), length 328) 10.20.0.1.67 > 10.20.0.2.68: BOOTP/DHCP, Reply, length 300, xid 0x7b8f9a6c, Flags [none]
	  Your-IP 10.20.0.2
	  Client-Ethernet-Address 10:10:10:10:10:ee
	  Vendor-rfc1048 Extensions
	    Magic Cookie 0x63825363
	    DHCP-Message Option 53, length 1: ACK
	    Server-ID Option 54, length 4: 10.20.0.1
	    Lease-Time Option 51, length 4: 43200
	    Subnet-Mask Option 1, length 4: 255.255.0.0
	    Default-Gateway Option 3, length 4: 10.20.0.1
	    Domain-Name-Server Option 6, length 4: 10.20.0.1
\end{Verbatim}

Для изучения работы протокола DHCP был запущен DHCP-сервер на маршрутизаторе \textbf{r1}. Данный сервер настроен так, чтобы клиенты получали статические IP-адреса, заданные в конфигурационном файле.

Конфигурационный файл данного DHCP-сервера, запущенного на маршрутизаторе \textbf{r1}:
\begin{Verbatim}
subnet 172.16.0.0 netmask 255.255.0.0 {}

subnet 10.10.0.0 netmask 255.255.0.0
{
  range 10.10.0.2 10.10.10.200;
  option routers 10.10.0.1;
  option domain-name-servers 10.10.0.1;
}

host ws11 {
    hardware ethernet 10:10:10:10:10:BA;
    fixed-address 10.10.1.1;
}

host ws12 {
    hardware ethernet 10:10:10:10:10:BB;
    fixed-address 10.10.2.1;
}

host ws13 {
    hardware ethernet 10:10:10:10:10:BC;
    fixed-address 10.10.3.1;
}


host ws14 {
    hardware ethernet 10:10:10:10:10:BD;
    fixed-address 10.10.4.1;
}
host s11 {
    hardware ethernet 10:10:10:10:20:AA;
    fixed-address 10.10.4.10;
}

host s12 {
    hardware ethernet 10:10:10:10:20:BB;
    fixed-address 10.10.4.20;
}

host s13 {
    hardware ethernet 10:10:10:10:20:CC;
    fixed-address 10.10.4.30;
}
\end{Verbatim}

Дамп собирался при помощи программы \textbf{tcpdump} на маршрутизаторе \textbf{r1}, запущенной с параметрами \textbf{-tenv -s 1000}.

Получение фиксированного адреса с помощью DHCP:
\begin{Verbatim}
10:10:10:10:20:cc > ff:ff:ff:ff:ff:ff, ethertype IPv4 (0x0800), length 342: (tos 0x10, ttl 128, id 0, offset 0, flags [none], proto UDP (17), length 328) 0.0.0.0.68 > 255.255.255.255.67: BOOTP/DHCP, Request from 10:10:10:10:20:cc, length 300, xid 0xb391540a, Flags [none]
	  Client-Ethernet-Address 10:10:10:10:20:cc
	  Vendor-rfc1048 Extensions
	    Magic Cookie 0x63825363
	    DHCP-Message Option 53, length 1: Discover
	    Requested-IP Option 50, length 4: 10.10.4.30
	    Parameter-Request Option 55, length 12: 
	      Subnet-Mask, BR, Time-Zone, Default-Gateway
	      Domain-Name, Domain-Name-Server, Option 119, Hostname
	      Netbios-Name-Server, Netbios-Scope, MTU, Classless-Static-Route
0e:ab:f8:0c:10:4b > 10:10:10:10:20:cc, ethertype IPv4 (0x0800), length 342: (tos 0x10, ttl 128, id 0, offset 0, flags [none], proto UDP (17), length 328) 10.10.0.1.67 > 10.10.4.30.68: BOOTP/DHCP, Reply, length 300, xid 0xb391540a, Flags [none]
	  Your-IP 10.10.4.30
	  Client-Ethernet-Address 10:10:10:10:20:cc
	  Vendor-rfc1048 Extensions
	    Magic Cookie 0x63825363
	    DHCP-Message Option 53, length 1: Offer
	    Server-ID Option 54, length 4: 10.10.0.1
	    Lease-Time Option 51, length 4: 43200
	    Subnet-Mask Option 1, length 4: 255.255.0.0
	    Default-Gateway Option 3, length 4: 10.10.0.1
	    Domain-Name-Server Option 6, length 4: 10.10.0.1
10:10:10:10:20:cc > ff:ff:ff:ff:ff:ff, ethertype IPv4 (0x0800), length 342: (tos 0x10, ttl 128, id 0, offset 0, flags [none], proto UDP (17), length 328) 0.0.0.0.68 > 255.255.255.255.67: BOOTP/DHCP, Request from 10:10:10:10:20:cc, length 300, xid 0xb391540a, Flags [none]
	  Client-Ethernet-Address 10:10:10:10:20:cc
	  Vendor-rfc1048 Extensions
	    Magic Cookie 0x63825363
	    DHCP-Message Option 53, length 1: Request
	    Server-ID Option 54, length 4: 10.10.0.1
	    Requested-IP Option 50, length 4: 10.10.4.30
	    Parameter-Request Option 55, length 12: 
	      Subnet-Mask, BR, Time-Zone, Default-Gateway
	      Domain-Name, Domain-Name-Server, Option 119, Hostname
	      Netbios-Name-Server, Netbios-Scope, MTU, Classless-Static-Route
0e:ab:f8:0c:10:4b > 10:10:10:10:20:cc, ethertype IPv4 (0x0800), length 342: (tos 0x10, ttl 128, id 0, offset 0, flags [none], proto UDP (17), length 328) 10.10.0.1.67 > 10.10.4.30.68: BOOTP/DHCP, Reply, length 300, xid 0xb391540a, Flags [none]
	  Your-IP 10.10.4.30
	  Client-Ethernet-Address 10:10:10:10:20:cc
	  Vendor-rfc1048 Extensions
	    Magic Cookie 0x63825363
	    DHCP-Message Option 53, length 1: ACK
	    Server-ID Option 54, length 4: 10.10.0.1
	    Lease-Time Option 51, length 4: 43200
	    Subnet-Mask Option 1, length 4: 255.255.0.0
	    Default-Gateway Option 3, length 4: 10.10.0.1
	    Domain-Name-Server Option 6, length 4: 10.10.0.1
\end{Verbatim}

\section{Использование VPN}

На маршрутизаторах \textbf{r1} и \textbf{r2} был настроен OpenVPN с помощью конфигурационного файла и в последствии запущен.

Таблица маршрутизации \textbf{r1} после запуска OpenVPN и работы RIP:
\begin{Verbatim}
10.100.100.2 dev tun0  proto kernel  scope link  src 10.100.100.1 
10.20.0.0/16 via 10.100.100.2 dev tun0  proto zebra  metric 2 
10.10.0.0/16 dev eth0  proto kernel  scope link  src 10.10.0.1 
172.16.0.0/16 dev eth1  proto kernel  scope link  src 172.16.1.3 
default via 172.16.1.2 dev eth1
\end{Verbatim}

Таблица маршрутизации \textbf{r2} после запуска OpenVPN и работы RIP:
\begin{Verbatim}
10.100.100.1 dev tun0  proto kernel  scope link  src 10.100.100.2 
10.20.0.0/16 dev eth0  proto kernel  scope link  src 10.20.0.1 
10.10.0.0/16 via 10.100.100.1 dev tun0  proto zebra  metric 2 
172.16.0.0/16 dev eth1  proto kernel  scope link  src 172.16.1.4 
default via 172.16.1.2 dev eth1
\end{Verbatim}

Назначенные IP-адреса на интерфейсах маршрутизатора \textbf{r1}:
\begin{Verbatim}
1: lo: <LOOPBACK,UP,LOWER_UP> mtu 16436 qdisc noqueue 
    inet 127.0.0.1/8 scope host lo
3: eth1: <BROADCAST,MULTICAST,UP,LOWER_UP> mtu 1500 qdisc pfifo_fast qlen 1000
    inet 172.16.1.3/16 brd 172.16.255.255 scope global eth1
4: eth0: <BROADCAST,MULTICAST,UP,LOWER_UP> mtu 1500 qdisc pfifo_fast qlen 1000
    inet 10.10.0.1/16 brd 10.10.255.255 scope global eth0
5: tun0: <POINTOPOINT,MULTICAST,NOARP,UP,LOWER_UP> mtu 1500 qdisc pfifo_fast qlen 100
    inet 10.100.100.1 peer 10.100.100.2/32 scope global tun0
\end{Verbatim}

Назначение IP-адресов на интерфейсах маршрутизатора \textbf{r2}:
\begin{Verbatim}
1: lo: <LOOPBACK,UP,LOWER_UP> mtu 16436 qdisc noqueue 
    inet 127.0.0.1/8 scope host lo
3: eth1: <BROADCAST,MULTICAST,UP,LOWER_UP> mtu 1500 qdisc pfifo_fast qlen 1000
    inet 172.16.1.4/16 brd 172.16.255.255 scope global eth1
4: eth0: <BROADCAST,MULTICAST,UP,LOWER_UP> mtu 1500 qdisc pfifo_fast qlen 1000
    inet 10.20.0.1/16 brd 10.20.255.255 scope global eth0
5: tun0: <POINTOPOINT,MULTICAST,NOARP,UP,LOWER_UP> mtu 1500 qdisc pfifo_fast qlen 100
    inet 10.100.100.2 peer 10.100.100.1/32 scope global tun0
\end{Verbatim}

Дамп RIP-сообщений на маршрутизаторе \textbf{r1}:
\begin{Verbatim}
ip: (tos 0x0, ttl 1, id 0, offset 0, flags [DF], proto UDP (17), length 52) 10.100.100.1.520 > 224.0.0.9.520: 
	RIPv2, Response, length: 24, routes: 1
	  AFI: IPv4:       10.10.0.0/16, tag 0x0000, metric: 1, next-hop: self
ip: (tos 0x0, ttl 1, id 0, offset 0, flags [DF], proto UDP (17), length 52) 10.100.100.2.520 > 224.0.0.9.520: 
	RIPv2, Response, length: 24, routes: 1
	  AFI: IPv4:       10.20.0.0/16, tag 0x0000, metric: 1, next-hop: self
ip: (tos 0x0, ttl 1, id 0, offset 0, flags [DF], proto UDP (17), length 52) 10.100.100.1.520 > 224.0.0.9.520: 
	RIPv2, Response, length: 24, routes: 1
	  AFI: IPv4:       10.10.0.0/16, tag 0x0000, metric: 1, next-hop: self
ip: (tos 0x0, ttl 1, id 0, offset 0, flags [DF], proto UDP (17), length 52) 10.100.100.2.520 > 224.0.0.9.520: 
	RIPv2, Response, length: 24, routes: 1
	  AFI: IPv4:       10.20.0.0/16, tag 0x0000, metric: 1, next-hop: self
\end{Verbatim}

Была проверена работа VPN с помощью проверки маршрута от \textbf{ws1}, находящегося в сети \textbf{10.20.0.0/16}, к которой подключен маршрутизатор \textbf{r2}, до \textbf{s11}, находящегося в сети \textbf{10.10.0.0/16}, к которой подключен маршрутизатор \textbf{r1}. 
\begin{Verbatim}
traceroute to 10.10.4.10 (10.10.4.10), 64 hops max, 40 byte packets
 1  10.20.0.1 (10.20.0.1)  12 ms  2 ms  2 ms
 2  10.100.100.1 (10.100.100.1)  2 ms  3 ms  2 ms
 3  10.10.4.10 (10.10.4.10)  11 ms  2 ms  3 ms
\end{Verbatim}

Как можно видеть из логов, маршрут проходит через сеть 10.100.100.1, что означает работу VPN.

\section{Правила фильтации пакетов и трансляции пдресов}

Где что дампим. 

\begin{Verbatim}
сценарий фильтрации
\end{Verbatim}

\begin{Verbatim}
iptables -L -nv
\end{Verbatim}

\begin{Verbatim}
iptables -L -nv -t nat
\end{Verbatim}

\section{Проверка трансляции SNAT}

Где что дампим.

\begin{Verbatim}
дамп SNAT в LAN (как вариант -i any tcp)
\end{Verbatim}

\begin{Verbatim}
дамп SNAT (снаружи)
\end{Verbatim}


\section{Проверка правил фильтрации}

Используем telnet.

\section{Проверка доступа к внутреннему серверу}

Используем telnet / веб-браузер на реальной машине. 
Должен быть виден DNAT и разрешённый доступ.

\end{document}
