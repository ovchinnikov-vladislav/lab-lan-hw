\documentclass[a4paper,12pt]{article}

\input{header.tex}

\title{Отчёт по лабораторной работе \\ <<Локальные сети>>}
\author{Овчинников Владислав Александрович}

\begin{document}

\maketitle

\tableofcontents

% Текст отчёта должен быть читаемым!!! Написанное здесь является рыбой.

\section{Получение адреса по DHCP}

Где что дампим (дампить с -tenv -s 1000).

\begin{Verbatim}
получение "случайного" адреса
\end{Verbatim}

Где что дампим.

\begin{Verbatim}
получение "фиксированого" адреса
\end{Verbatim}


\section{Использование VPN}

\begin{Verbatim}
ip r на маршрутизаторе после VPN и работы RIP
\end{Verbatim}

\begin{Verbatim}
ip -4 a  на маршрутизаторе
\end{Verbatim}

\begin{Verbatim}
просшулка сообщений RIP на tun0
\end{Verbatim}

Проверка работы VPN

\begin{Verbatim}
Трейс с ws21 до s11
\end{Verbatim}

\section{Правила фильтации пакетов и трансляции пдресов}

Где что дампим. 

\begin{Verbatim}
сценарий фильтрации
\end{Verbatim}

\begin{Verbatim}
iptables -L -nv
\end{Verbatim}

\begin{Verbatim}
iptables -L -nv -t nat
\end{Verbatim}

\section{Проверка трансляции SNAT}

Где что дампим.

\begin{Verbatim}
дамп SNAT в LAN (как вариант -i any tcp)
\end{Verbatim}

\begin{Verbatim}
дамп SNAT (снаружи)
\end{Verbatim}


\section{Проверка правил фильтрации}

Используем telnet.

\section{Проверка доступа к внутреннему серверу}

Используем telnet / веб-браузер на реальной машине. 
Должен быть виден DNAT и разрешённый доступ.

\end{document}
